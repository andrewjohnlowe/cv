\begin{flushleft}
{\bf {ACADEMIC EMPLOYMENT}}\\ %% other employment/other experience/other work? professional career, research experience/other experience
\end{flushleft}
\vspace{\longtabletopsepspecial}
\begin{longtable}{p{\firstcolumnwidth}p{\secondcolumnwidth}}
\toprule
\firstcolumndata{September}& {\bf Scientific Fellow, \htmladdnormallink{ALICE Experiment}{http://aliceinfo.cern.ch/Public/Welcome.html}}\\
\firstcolumndata{2013--present}& \htmladdnormallink{Wigner Research Centre for Physics, Hungarian Academy of Sciences}{http://www.rmki.kfki.hu/en}, Budapest, Hungary\\
& \\
& Development of jet substructure, superstructure, and jet-by-jet tagging algorithms for ALICE offline physics analysis.\\
& \\
& The goal of this work is to find new observables that can be used to extract more information from hadronic events, and from the jets individually, than would have been possible otherwise, thereby improving discovery reach in searches for new physics and enhancing precision tests of QCD at the energy frontier.\\
& \\
%#Data Analysis Work
%#● Working on the ATLAS experiment on the Large Hadron Collider at CERN
%#● Awarded distinguished 1 year US ATLAS fellowship by Argonne National Laboratory
%#● Completed and published 12 complex statistical analysis projects
%#● Projects each directly used 10-200 TB of data
%#● Increased sensitivity of hundreds of analyses 10-30% through improvements and calibrations of machine learning data analysis algorithms
%#● Promoted to data analysis management positions
%#● Constantly ran data analysis and visualization using Python, C++, linux batch systems, and distributed analysis software similar to Hadoop
% Institute for Particle and Nuclear Physics, Wigner Research Centre for Physics, Hungarian Academy of Sciences
% At Wigner: hired by DG himself. Post created for me. Contributions to HEP masterclass? Big Data work?
\firstcolumndata{April 2010--}& {\bf Postdoctoral Fellow, Deputy Team Leader, \htmladdnormallink{ATLAS Experiment}{http://atlas.ch/}}\\
\firstcolumndata{October 2012}& \htmladdnormallink{California State University, Fresno, USA}{http://www.fresnostate.edu/csm/physics/} (based at CERN)\secondcolumndata{, 2010--2012} \\
& \\
& Development of jet substructure, superstructure, and jet-by-jet tagging algorithms for ATLAS Trigger and offline physics analysis:\\
&\\
& Systematically investigated the potential benefit of hundreds of hadronic event shape variables and jet shape variables for a variety of Higgs and new physics searches (SM Higgs $WH \rightarrow \ell\nu\,b\bar{b}$, Hidden Valley Higgs $H\rightarrow\pi_\mathrm{v}\pi_\mathrm{v}\rightarrow{b\bar{b}}{b\bar{b}}$, hadronic decays of colour octet scalars and excited $W$ bosons) and for quark/gluon tagging. Variables include my own inventions and those compiled from a thorough survey of relevant literature. Several new promising variables were found that provide significant additional discrimination power against QCD background. They have the potential for significant impact on many new physics searches. Presentations at the ATLAS Higgs $H\rightarrow{b\bar{b}}$, Higgs properties, SM jet physics, boosted objects, jet trigger, and exotics working groups.\\
%& Systematically investigated the potential benefit of hundreds of different hadronic event shape variables and jet shape variables for a variety of new physics searches (SM Higgs $WH \rightarrow \ell\nu\,b\bar{b}$, Hidden Valley Higgs $H\rightarrow\pi_\mathrm{v}\pi_\mathrm{v}\rightarrow{b\bar{b}}{b\bar{b}}$, hadronic decays of colour octet scalars and excited $W$ bosons) and for quark/gluon tagging. Variables include my own inventions and those compiled from a thorough survey of relevant literature. Several new promising variables were found that provide significant additional discrimination power against QCD background. They have the potential for significant impact on many new physics searches. Presentations at the ATLAS Higgs $H\rightarrow{b\bar{b}}$, Higgs properties, SM jet physics, boosted objects, jet trigger, and exotics working groups.\\
&\\
& Developed a trigger algorithm to demonstrate the technical feasibility of using colour-connection variables in the jet trigger for tighter selection. This was the first attempt to use such techniques in the ATLAS trigger.\\
%%, as an alternative to using higher energy thresholds and prescaling.\\ Given that the jet triggers are amongst those that will be prescaled aggressively as instantaneous luminosity increases, it is urgent to consider using tighter cuts as an alternative to using higher energy thresholds and prescaling. 
%% Have investigated the potential benefit of hundreds of different event shapes and jet shapes for both Higgs and exotics offline analyses
%% Many of these are my own inventions, the rest are those that I encountered during a thorough survey of the relevant papers and literature
%% Long-term goal was to develop a new trigger for the jet slice that uses event shapes or jet shapes that have come to light in the past few years
%% ~400 event shapes, ~600 jet shapes
%% Preliminary results shows that these variables have limited dependencies on event kinematics and pile-up conditions
%% I am keen to investigate the potential of the jet trigger algorithms to boost selection efficiency for interesting physics signatures by 

%% Talks at the ATLAS Higgs $H\rightarrow\{b\bar{b}}$, Higgs Properties, SM Jet Physics, Boosted Objects, Jet Trigger, and Exotics Working Groups from 2009 to present.
&\\ %% why is vertex trigger important -- strassler. interface
%& Co-supervised a PhD student in the development of a dijet trigger algorithm for the 2012 trigger menu at the request of the exotic dijets group, who required the algorithm for running an unprescaled trigger in the phase space region $m_{\mathrm{jj}}>2\TeV$, $\left|y_{1} - y_{2}\right|/2<1.7$. Provided help and orientation, in the form of weekly group meetings and one-on-one mentoring, to CSU students based at CERN. Initiated and oversaw the enrollment of CSU students for Control Room shifts.\\
& Co-supervised a PhD student in the development of a dijet trigger algorithm for the 2012 trigger menu that was essential for running an unprescaled trigger in the phase space region $m_{\mathrm{jj}}>2\TeV$, $\left|y_{1} - y_{2}\right|/2<1.7$. Provided help and orientation, in the form of weekly group meetings and one-on-one mentoring, to CSU students based at CERN. Initiated and oversaw the enrollment of CSU students for Control Room shifts.\\
& \\
& Contributed to the writing of bids for research grants: \NSF \EPP core grant (\$\numprint[US]{1033000}) and \NSF \MRI grant (\$\numprint[US]{620000}).\\ 
%& Supervised JGU Mainz PhD student Oliver Endner in the development of a dijet trigger for the 2012 trigger menu that is fully efficient for events with $m_{\mathrm{jj}}>2\TeV$ and $\left|y^{\ast}\right|=\left|y_{1} - y_{2}\right|$/2 $<$ 1.7. The algorithm was requested by the exotic dijets group as crucial for running an unprescaled trigger in that region of phase space.\\
&\\
\firstcolumndata{February 2008--}& {\bf Postdoctoral Fellow, \htmladdnormallink{ATLAS Experiment}{http://atlas.ch/}}\\
\firstcolumndata{August 2009}& \htmladdnormallink{Indiana University, USA}{http://www.physics.indiana.edu/} (based at CERN)\secondcolumndata{, 2008--2009}\\
& \\
& Collaborated with the $b$-jet trigger group and the exotic long-lived particles group:\\
&\\
& Developed a trigger algorithm for enabling secondary vertex based $b$-tagging to be performed at the \EF for the first time. Made comparisons of online performance with that of offline vertexing tools to optimise algorithm parameters. Achieved excellent agreement between online and offline vertexing performance. Demonstrated that the trigger algorithm runs safely within the time constraints of the \EF. The algorithm has run during data-taking since 2010 and has been used to actively select events at the trigger level since 2012.\\%cavaliere:2012
&\\ %% why is vertex trigger important -- strassler. interface
& Worked on improving and adapting the trigger algorithm and the offline vertexing tools to enable selection of events with highly-displaced vertices due to the decay of long-lived neutral particles that arise in new physics models, such as Hidden Valley and R-parity violating \SUSY. The trigger algorithm and offline vertexing were each run on simulated Hidden Valley $H\rightarrow\pi_\mathrm{v}\pi_\mathrm{v}\rightarrow{b\bar{b}}{b\bar{b}}$ events to identify deficiencies and to explore distributions of relevant discriminant variables to optimise cuts for online selection and offline physics analysis.
%%& Performed trigger software validation duties for the $b$-jet trigger slice, which involved diagnosing problems, bug reporting, tracking problems to solution, and representation in software validation meetings.\\
%
%& Developed a trigger algorithm for enabling secondary vertex based $b$-tagging to be performed at the \EF. Demonstrated that this trigger algorithm runs safely within the the times constraints of the \EF and is now included in the physics trigger menu. Work is proceeding on comparisons of the online implementation with the performance of offline secondary vertex algorithms to optimise algorithm parameters. Performed trigger software validation duties for the $b$-jet trigger slice, which involved diagnosing problems, bug reporting, and tracking problems to solution. Recent focus has been on examination of the feasibility of adapting the algorithm to enable triggering on events with highly-displaced vertices due to long-lived neutral particles that arise in new physics models such as Hidden Valley models and R-parity violating \SUSY. The trigger algorithm has been run on simulated Hidden Valley $H\rightarrow\pi_\mathrm{v}\pi_\mathrm{v}\rightarrow{b\bar{b}}{b\bar{b}}$ events to identify deficiencies and to explore distributions of relevant discriminant variables to optimise selection.\\
\end{longtable}
