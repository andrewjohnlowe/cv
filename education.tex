%\begin{longtable}{p{0.15\textwidth}p{0.85\textwidth}}
\begin{longtable}{p{\firstcolumnwidth}p{\secondcolumnwidth}}
\textcolor{color1}{\bf EDUCATION} & \\
\arrayrulecolor{color1}
\toprule
\firstcolumndata{2001--2008} & {\bf PhD Particle Physics}\\
& {\it \htmladdnormallink{Royal Holloway, University of London, UK}{http://www.rhul.ac.uk/} (including 17 months at \htmladdnormallink{CERN}{http://cern.ch/}, Switzerland)}\secondcolumndata{, 2008}\\
& \htmladdnormallink{Thesis title: \emph{Performance and robustness studies of the trigger for the ATLAS experiment}}{http://cdsweb.cern.ch/record/1120800?ln=en}\\
%& \\
%& Demonstrated that the electron trigger is robust against the effects of both dead cells and dead \FEBs in the electromagnetic calorimeter; determined that the trigger efficiency for benchmark physics channels $\W\rightarrow\enu$, $\Z\rightarrow\ee$ and $H\rightarrow\ZZstar\rightarrow\eeee$ decreases by no more than 1\% per 1\% increase in dead cells or dead \FEBs.\\
%%&\\
%%& Played a major role in the development of an object-oriented analysis tool used for electron trigger performance studies.\\
&\\
& Played a major role in the development of the core software and algorithms in \Cplusplus for a real-time multi-stage cascade classifier that filters and reduces the collision event data rate from 60\TBs to a manageable 300\MBs that can be written to permanent storage. Performed detailed time profiling of the core software and devised improvements that made it 8 times faster, thus meeting a critical requirement of the system. Wrote \htmladdnormallink{software that was used in the discovery of the Higgs boson}{https://arxiv.org/abs/0901.0512}.\\
%& Spent two years working in a team developing the \HLT software framework. Made numerous key contributions to the \HLT algorithm steering mechanism during the early stages of its development. Developed a \htmladdnormallink{trigger algorithm}{http://indico.cern.ch/conferenceDisplay.py?confId=a022027} widely used as an exemplar by algorithm developers integrating their code with the \HLT software framework. Played a major role in the development of an object-oriented analysis tool used for electron trigger performance studies.\\
%& Spent over a year at CERN working in a team developing the \HLT software framework. Made numerous key contributions to the \HLT algorithm steering mechanism. Developed a \htmladdnormallink{trigger algorithm}{http://indico.cern.ch/conferenceDisplay.py?confId=a022027} widely used as an exemplar by algorithm developers integrating their code with the \HLT software framework. Played a major role in the development of an object-oriented analysis tool used for electron trigger performance studies.\\
&\\
%& Performed detailed time profiling of \HLT code and devised improvements that made it 8 times faster, thus meeting a critical requirement of the trigger.\\
%&\\
%% The team benefited from my rigorous approach and attention to detail. 

%% Was able to apply the skills acquired in courses at RAL develop my \Cplusplus programming experience.

%% I was most proud of my ability to solve technical problems that other people were stuck on.

%% Studied the effect of unresponsive (``dead'') electromagnetic calorimeter cells and on-detector electronics boards on electron trigger efficiencies with respect to the standard physics channels $\W\rightarrow\enu$, $\Z\rightarrow\ee$ and $H\rightarrow\ZZstar\rightarrow\eeee$. Demonstrated that the electron trigger efficiency decreases by no more than 1\% per 1\% increase in dead calorimeter cells and electronics boards.

%Performed detailed time profiling of code and devised improvements that made it 8 times faster.

%% Developed a trigger algorithm widely used by algorithm developers as an exemplar for integrating their code with the \HLT software framework.

%% Author of a trigger algorithm used as an exemplar by algorithm developers integrating their code with the \HLT core software.
%% & \\
%% & \\
%% \pagebreak
\firstcolumndata{2000--2001} & {\bf MSc Particle Physics}\\
& {\it \htmladdnormallink{Royal Holloway, University of London, UK}{http://www.rhul.ac.uk/}}\secondcolumndata{, 2001}\\
& \htmladdnormallink{Thesis title: \emph{Light Higgs $(H \rightarrow b \bar{b})$ at the LHC}}{http://cdsweb.cern.ch/record/1191166?ln=en}\\
& \\
& Investigated the search potential of the $H \rightarrow b\bar{b}$ decay channel for a light Higgs boson using the \htmladdnormallink{ATLAS detector}{http://atlas.ch/} at CERN. First data-mining analysis of this type to be performed entirely in \Cplusplus.\\ % mentioning that this was a data mining task
& \\
\firstcolumndata{1993--1996} & {\bf BSc (Hons) Physics}\\
& {\it \htmladdnormallink{Royal Holloway, University of London, UK}{http://www.rhul.ac.uk/}}\secondcolumndata{, 1996}\\
%& \\
& Final year project: \emph{Primordial nucleosynthesis}\\ % using Maple
%& \\
%\firstcolumndata{1989--1993} & \htmladdnormallink{St. Mary's College, Hull, UK}{http://www.stmaryscollegehull.co.uk/}\secondcolumndata{, 1989--1993}\\
%& Four Advanced Levels (A-Levels), one Advanced Supplementary Level (AS-Level), ten GCSEs.
\end{longtable}
