\begin{flushleft}
{\bf COMPUTING SKILLS}\\
\end{flushleft}
\vspace{\longtabletopsepspecial}
\begin{longtable}{p{0.33\textwidth}p{0.33\textwidth}p{0.33\textwidth}}
\toprule
\vspace{2\longtabletopsep}
\begin{ilist} % 1
\noitem {\bf Programming languages:}
\item {\sc C++/STL} \emph{(13 years' experience)}
\item R
\item Octave
%\item SQL
\item Python
\item Unix shell \mbox{scripting}
\item awk/sed
\item \mbox{\sc FORTRAN}
\item Microsoft Excel VBA macros
\item Pascal
\item Delphi
\item Maple
\item BASIC
\end{ilist} &
\vspace{2\longtabletopsep}
\begin{ilist} % 2
\noitem {\bf Software development:}
\item Emacs
\item Sublime Text
\item RStudio
\item UML
\item Borland Together
\item {\sc LXR}
\item Git/GitHub % Add link to my git repo here?
%\item {\sc \htmladdnormallink{CMT}{http://www.cmtsite.org/}}
\item CVS
\item \SVN
\item Valgrind/Callgrind/KCachegrind
\item GDB
\item Savannah
\end{ilist} &
\vspace{2\longtabletopsep}
\begin{ilist} % 3
\noitem {\bf Documentation:}
\item \LaTeX\ \emph{(13 years' experience)}
\item Markdown
\item {\sc HTML}
\item {\sc XML}
\item {\sc CSS}
\item TWiki
\item Doxygen
\end{ilist} \\
\vspace{2\longtabletopsep}
\begin{ilist} % 1
\noitem {\bf HEP software:}
\item {\sc \htmladdnormallink{ROOT}{http://root.cern.ch/}} \emph{(12 years' experience)}
\item {\sc \htmladdnormallink{PAW}{http://paw.web.cern.ch/paw/}}
\item {\sc \htmladdnormallink{PYTHIA}{http://www.thep.lu.se/~torbjorn/Pythia.html}}
%\item {\htmladdnormallink{Atlfast}{http://www.hep.ucl.ac.uk/atlas/atlfast/}}
\end{ilist} &
\vspace{2\longtabletopsep}
\begin{ilist}  % 2
\noitem {\bf Office suites:}
\item Microsoft Office
\item LibreOffice/OpenOffice
\end{ilist} &
\vspace{2\longtabletopsep}
\begin{ilist} % 3
\noitem {\bf Operating systems:}
\item Unix/Linux\,\emph{(13 years' experience)} % start 2000
\item Microsoft Windows
\item VAX/VMS
\end{ilist}
%\multicolumn{2}{l}{Extensive experience of trigger systems at collider experiments, grid computing, data-mining and .}\\
%& \\
%{\bf Programming languages:} & {\sc C/C++/STL} \emph{(12 years' experience)}~\footersymbol~Python~\footersymbol~Unix shell \mbox{scripting}~\footersymbol~awk/sed~\footersymbol~\mbox{\sc FORTRAN}~\footersymbol~Pascal~\footersymbol~Delphi~\footersymbol~Maple~\footersymbol~BASIC\\[0.5\rowgap]
%{\bf Software development:} & UML~\footersymbol~Borland Together~\footersymbol~{\sc LXR}~\footersymbol~{\sc \htmladdnormallink{CMT}{http://www.cmtsite.org/}}~\footersymbol~CVS~\footersymbol~Subversion~\footersymbol~Valgrind~\footersymbol~GDB\\[0.5\rowgap]
%{\bf Operating systems:} & Unix/Linux \emph{(12 years' experience)}~\footersymbol~Microsoft Windows~\footersymbol~VAX/VMS\\[0.5\rowgap]
%{\bf Documentation:} & \LaTeX\ \emph{(12 years' experience)}~\footersymbol~{\sc HTML}~\footersymbol~{\sc XML}~\footersymbol~{\sc CSS}~\footersymbol~TWiki~\footersymbol~Doxygen\\[0.5\rowgap]
%{\bf Office suites:} & Microsoft Office~\footersymbol~LibreOffice/OpenOffice\\[0.5\rowgap]
%{\bf Data analysis:} & {\sc \htmladdnormallink{ROOT}{http://root.cern.ch/}} \emph{(11 years' experience)}\\[0.5\rowgap]
%{\bf Monte Carlo simulation:} & {\sc \htmladdnormallink{PYTHIA}{http://www.thep.lu.se/~torbjorn/Pythia.html}}~\footersymbol~{\htmladdnormallink{Atlfast}{http://www.hep.ucl.ac.uk/atlas/atlfast/}}
\end{longtable}
\vspace{2\longtabletopsepspecial}
\begin{flushleft}
Other: object-oriented analysis and design, grid computing, {\sc C++} template metaprogramming, CPU and time profiling, memory debugging, code optimisation, advanced floating point computing, data recovery. Over 20 years' programming experience in various languages.
\end{flushleft}
%\vspace{2\longtabletopsep}
% Markup languages: HTML, CSS, XML

%XML, experience with big data, memory, CPU profiling, metaprogramming,  (Bash, AWK, sed)

%Extensive experience of data-acquisition systems, data mining and statistical analysis of huge multivariate datasets and Monte Carlo simulation techniques.

%Personal qualities:

%• Self-starter, willing to take initiative to creatively solvechallenges

%• Resourceful

%• Ability to work independently and as part of a team

%• Excellent oral and written communication skills
%• Ability to thrive and enjoy a fast-paced environment

%experience in C++ programming, algorithms, data structures, and OO design 

% analysed simulated datasets to determine potential measurments at future experiments
% effectively communicated through journal papers, internal documentation and weekly presnetations as well as presentations at international conferences and university colloquia
% led in the design, implentation and operation of networked VME data acquistion for cutting-edge particle physicis experiment [at international scientific laboroatory]. 
%specialised in the design, implementation and support of complex [software] systems, as well as effective copmmunication with critcal audiances
%I am primarily a physics researcher within the field of experimental particle physics. As such I am highly trained in statistical data analysis and implementations thereof in various programming environments.

%Specialties: From a physics poit of view I am a physics analyst and a toolkit developer (C++). From an industry point of view I can be seen as a software developer (C++,f77,...) with expertise in working in extremely complex environments.
