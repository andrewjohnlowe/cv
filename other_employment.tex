\begin{flushleft}
{\bf OTHER EMPLOYMENT}\\ %% other employment/other experience/other work? professional career, research experience/other experience
\end{flushleft}
\vspace{\longtabletopsepspecial}
%%\begin{tabitemize}
%\begin{longtable}{p{0.15\textwidth}p{0.85\textwidth}}
\begin{longtable}{p{\firstcolumnwidth}p{\secondcolumnwidth}}
\toprule
\firstcolumndata{1998--2000} & {\bf {Assistant Research Scientist}}\\
& \htmladdnormallink{Centre for Time Metrology}{http://www.npl.co.uk/server.php?show=nav.348}, \htmladdnormallink{\NPL}{http://www.npl.co.uk/}, UK\secondcolumndata{, 1998--2000}\\
%% & Assistant Research Scientist\\{\bf \htmladdnormallink{\NPL}{http://www.npl.co.uk/}}\\
& \\
& Provided technical and administrative support to a range of key activities relating to the maintenance and dissemination of the UK's national time scale:\\
&\\
& Developed software to process data from \NPL's geodetic-quality GPS receivers. Collected and analysed GPS timing data. My work underpinned two projects, resulting in the publication of three time metrology research papers to which I made a substantial contribution.\\%%\footnote{The results of this work are directly pertinent to long-baseline neutrino time-of-flight measurements.\vspace{-1.5\baselineskip}}.\\
% & Substantial contribution to three time metrology research papers.\\
&\\
& Maintained the caesium atomic clocks and time scale generation equipment that form the UK's national time scale. Compiled, processed and submitted data used to compute the international time scale, Coordinated Universal Time (UTC). Operated a satellite earth station. Acted as \NPL's proxy for authorising changes (such as the start and end of daylight saving time) to the \htmladdnormallink{MSF}{http://www.npl.co.uk/server.php?show=ConWebDoc.998} 60\kHz national time radio signal, which synchronises a considerable number of radio-controlled clocks across the UK. Prepared technical manuals covering key procedures, in accordance with \NPL's ISO 9001 QA certification.\\
&\\
& Published monthly measurements bulletins and manned an enquiry service that received hundreds of telephone calls, letters, faxes and emails each year from the public.\\
%RF and Microwave Free Field Section (March 2000 - September 2000):
%I provided administrative support to NPL's antenna and field probe calibration service. This involved dealing with customers both face-to-face and over the telephone, preparing quotations, receiving and despatching customer's artefacts, scheduling calibration work, maintaining the contact database, and producing calibration certificates. I also assisted in the calibration of wire antennas used for Electromagnetic Compatibility (EMC) conformance testing.
%%&\\
%%& Substantial contribution to three time metrology research papers.\\
%research papers.\\
%% My work underpinned two projects, resulting in the publication of three papers~\cite{lowe:1998,lowe:1999,lowe:2000}.\\
& \\
\firstcolumndata{1996--1998} & \secondcolumndata{{\bf {Various temporary positions}}}\firstcolumndata{Various temporary positions}, including VCR retuning engineer for \htmladdnormallink{Channel 5 TV}{http://en.wikipedia.org/wiki/Five_tv}\secondcolumndata{, 1996--1998}\firstcolumndata{.}
\end{longtable}
%NPL is the UK's centre for precise time and frequency measurement.

%We operate the group of atomic clocks that form the national time scale, known as UTC(NPL).

%The time scale provides the reference for a range of services that disseminate the Time from NPL. These services include the MSF radio time signal, which synchronises a considerable number of radio-controlled clocks across the UK.

%The clocks at NPL are compared with those at other national timing institutes and contribute to the international time scale using highly accurate time and frequency transfer methods. NPL is a world leader in the analysis of clock and time transfer data, and is making a key contribution to the development of the timing infrastructure of the Galileo satellite navigation system.
