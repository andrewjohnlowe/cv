\begin{flushleft}
\textcolor{color1}{\bf {EMPLOYMENT}}\\ %% other employment/other experience/other work? professional career, research experience/other experience
\end{flushleft}
\vspace{\longtabletopsepspecial}
\begin{longtable}{p{\firstcolumnwidth}p{\secondcolumnwidth}}
\arrayrulecolor{color1}
\toprule
\firstcolumndata{Jun. 2017--}& {\bf Data Scientist/Business Analyst}\\
\firstcolumndata{present}& {\it \htmladdnormallink{EPAM Systems Inc., Hungary}{http://www.epam.com}}\\
& \\
& \emph{Responsibilities:}
Convert large volumes of structured and unstructured customer data using advanced analytical solutions.
Use and fit mathematical and econometric models, and develop descriptive and predictive models that deliver better decisions.
Turn analysed data into actionable insights and business value.
Create high-quality data visualisations in cooperation with business analysts.
Communicate effectively with other departments (product managers, engineers) to discuss complex data-driven findings and technical specifications.
Continuously develop the knowledge base within the data scientist team and participate in common brainstorming sessions.\\
%& \\
%& \emph{Project participation:} Development of a cloud-hosted chatbot to process customer service requests for a major investment bank. {Role:} Information Architect. {Technologies used:} IBM Watson Conversation, JSON.\\
& \\
\firstcolumndata{Sept. 2013--}& {\bf Scientific Research Fellow}\\
\firstcolumndata{May 2017}& {\it \htmladdnormallink{Wigner Research Centre for Physics, Hungarian Academy of Sciences, Hungary}{http://www.rmki.kfki.hu/en}}\\
& \\
& Performed statistical data analysis for the \htmladdnormallink{ALICE experiment}{http://aliceinfo.cern.ch/Public/Welcome.html} at \htmladdnormallink{CERN}{https://home.cern}, which recreates conditions that are believed to have existed a fraction of a second after the Big Bang. Used state-of-the-art machine learning to develop predictive classification algorithms for recognising particles based on their decay properties.\\
& \\
& Conducted the first-ever particle physics data analysis performed entirely in the R statistical programming language. Devised a novel fast data-driven feature selection method that identifies the variables with the best predictive power for a given classification or regression task. Developed pattern recognition algorithms that promise to improve discovery reach in searches for new particles at CERN and beyond.\\
& \\
& Pioneered implementation of reproducible research by writing the first-ever fully reproducible particle physics analysis paper. Founded the \mbox{ALICE} Statistics and Machine Learning Working Group. Co-organiser of the first \htmladdnormallink{CERN workshop}{http://www.nature.com/news/artificial-intelligence-called-in-to-tackle-lhc-data-deluge-1.18922} dedicated to the use of machine learning in particle physics. Taught machine learning \htmladdnormallink{tutorials}{https://www.linkedin.com/pulse/introduction-machine-learning-particle-physics-andrew-lowe?trk=prof-post} at E\"{o}tv\"{o}s Lor\'{a}nd University and CERN. Engaged with local data science community via \htmladdnormallink{public outreach talks}{https://www.linkedin.com/pulse/machine-learning-particle-physics-using-r-andrew-lowe?trk=prof-post} and conference presentations.\\
& \\
%#Data Analysis Work
%#● Working on the ATLAS experiment on the Large Hadron Collider at CERN
%#● Awarded distinguished 1 year US ATLAS fellowship by Argonne National Laboratory
%#● Completed and published 12 complex statistical analysis projects
%#● Projects each directly used 10-200 TB of data
%#● Increased sensitivity of hundreds of analyses 10-30% through improvements and calibrations of machine learning data analysis algorithms
%#● Promoted to data analysis management positions
%#● Constantly ran data analysis and visualization using Python, C++, linux batch systems, and distributed analysis software similar to Hadoop
% Institute for Particle and Nuclear Physics, Wigner Research Centre for Physics, Hungarian Academy of Sciences
% At Wigner: hired by DG himself. Post created for me. Contributions to HEP masterclass? Big Data work?
\firstcolumndata{Apr. 2010--}& {\bf Postdoctoral Fellow, Deputy Team Leader}\\
\firstcolumndata{Oct. 2012}& {\it \htmladdnormallink{California State University, Fresno, USA}{http://www.fresnostate.edu/csm/physics/} (based at CERN)}\secondcolumndata{, 2010--2012} \\
& \\
& \htmladdnormallink{Systematically investigated}{https://indico.cern.ch/event/135893/contributions/1358845/attachments/109246/155487/lowe-2011-04-19.pdf} the potential benefit of hundreds of different predictor variables for a range of analyses using Monte Carlo simulations written in \Cplusplus. Several new variables were found that provide significant improvements in sensitivity for a variety of Higgs boson and new particle searches.\\
%& Systematically investigated the potential benefit of hundreds of different hadronic event shape variables and jet shape variables for a variety of new physics searches (SM Higgs $WH \rightarrow \ell\nu\,b\bar{b}$, Hidden Valley Higgs $H\rightarrow\pi_\mathrm{v}\pi_\mathrm{v}\rightarrow{b\bar{b}}{b\bar{b}}$, hadronic decays of colour octet scalars and excited $W$ bosons) and for quark/gluon tagging. Variables include my own inventions and those compiled from a thorough survey of relevant literature. Several new promising variables were found that provide significant additional discrimination power against QCD background. They have the potential for significant impact on many new physics searches. Presentations at the ATLAS Higgs $H\rightarrow{b\bar{b}}$, Higgs properties, SM jet physics, boosted objects, jet trigger, and exotics working groups.\\
%&\\
%& Developed a trigger algorithm to demonstrate the technical feasibility of using colour-connection variables in the jet trigger for tighter selection. This was the first attempt to use such techniques in the ATLAS trigger.\\
%& \\
%%, as an alternative to using higher energy thresholds and prescaling.\\ Given that the jet triggers are amongst those that will be prescaled aggressively as instantaneous luminosity increases, it is urgent to consider using tighter cuts as an alternative to using higher energy thresholds and prescaling. 
%% Have investigated the potential benefit of hundreds of different event shapes and jet shapes for both Higgs and exotics offline analyses
%% Many of these are my own inventions, the rest are those that I encountered during a thorough survey of the relevant papers and literature
%% Long-term goal was to develop a new trigger for the jet slice that uses event shapes or jet shapes that have come to light in the past few years
%% ~400 event shapes, ~600 jet shapes
%% Preliminary results shows that these variables have limited dependencies on event kinematics and pile-up conditions
%% I am keen to investigate the potential of the jet trigger algorithms to boost selection efficiency for interesting physics signatures by 

%% Talks at the ATLAS Higgs $H\rightarrow\{b\bar{b}}$, Higgs Properties, SM Jet Physics, Boosted Objects, Jet Trigger, and Exotics Working Groups from 2009 to present.
%&\\ %% why is vertex trigger important -- strassler. interface
%& Co-supervised a PhD student in the development of a dijet trigger algorithm for the 2012 trigger menu at the request of the exotic dijets group, who required the algorithm for running an unprescaled trigger in the phase space region $m_{\mathrm{jj}}>2\TeV$, $\left|y_{1} - y_{2}\right|/2<1.7$. Provided help and orientation, in the form of weekly group meetings and one-on-one mentoring, to CSU students based at CERN. Initiated and oversaw the enrollment of CSU students for Control Room shifts.\\
%& Co-supervised a PhD student in the development of a dijet trigger algorithm for the 2012 trigger menu that was essential for running an unprescaled trigger in the phase space region $m_{\mathrm{jj}}>2\TeV$, $\left|y_{1} - y_{2}\right|/2<1.7$.
%& Supervised and provided help and orientation, in the form of weekly group meetings and one-on-one mentoring, to master's degree students based at CERN. Initiated and oversaw the enrollment of students for ATLAS Control Room shifts.\\
%& \\
%& Contributed to the writing of bids for research grants: \NSF \EPP core grant (\$\numprint[US]{1033000}) and \NSF \MRI grant (\$\numprint[US]{620000}).\\ 
%& Supervised JGU Mainz PhD student Oliver Endner in the development of a dijet trigger for the 2012 trigger menu that is fully efficient for events with $m_{\mathrm{jj}}>2\TeV$ and $\left|y^{\ast}\right|=\left|y_{1} - y_{2}\right|$/2 $<$ 1.7. The algorithm was requested by the exotic dijets group as crucial for running an unprescaled trigger in that region of phase space.\\
&\\
\firstcolumndata{Feb. 2008--}& {\bf Postdoctoral Fellow}\\
\firstcolumndata{Aug. 2009}& {\it \htmladdnormallink{Indiana University, USA}{http://www.physics.indiana.edu/} (based at CERN)}\secondcolumndata{, 2008--2009}\\
& \\
%& Collaborated with the $b$-jet trigger group and the exotic long-lived particles group:\\
%&\\
& Developed an algorithm in \Cplusplus for real-time particle identification in streaming data at an input rate of 1\GBs. Optimised algorithm parameters and achieved excellent performance. This algorithm underpins a large part of the ATLAS experiment's physics programme by providing the data used for many analyses. It has been used for data-taking since 2010 and has currently processed 70\PB and recorded 3\PB for subsequent analysis. %http://iopscience.iop.org/article/10.1088/1742-6596/513/1/012004/pdf doi:10.1088/1742-6596/513/1/012004 (1GB/s = 1.5 MB * 900 Hz) (Run 1 ~ 609 days, Run 2 to 2017 ~ 586 = total ~1*10^8 seconds. 1*10^8 seconds * 1.1 GB/s = 110 PB. Assuming data-taking efficiency during this time, 55 PB)
\\%cavaliere:2012
%&\\ %% why is vertex trigger important -- strassler. interface
%& Worked on improving and adapting the trigger algorithm and the offline vertexing tools to enable selection of events with highly-displaced vertices due to the decay of long-lived neutral particles that arise in new physics models, such as Hidden Valley and R-parity violating \SUSY. The trigger algorithm and offline vertexing were each run on simulated Hidden Valley $H\rightarrow\pi_\mathrm{v}\pi_\mathrm{v}\rightarrow{b\bar{b}}{b\bar{b}}$ events to identify deficiencies and to explore distributions of relevant discriminant variables to optimise cuts for online selection and offline physics analysis.\\
%%& Performed trigger software validation duties for the $b$-jet trigger slice, which involved diagnosing problems, bug reporting, tracking problems to solution, and representation in software validation meetings.\\
%
%& Developed a trigger algorithm for enabling secondary vertex based $b$-tagging to be performed at the \EF. Demonstrated that this trigger algorithm runs safely within the the times constraints of the \EF and is now included in the physics trigger menu. Work is proceeding on comparisons of the online implementation with the performance of offline secondary vertex algorithms to optimise algorithm parameters. Performed trigger software validation duties for the $b$-jet trigger slice, which involved diagnosing problems, bug reporting, and tracking problems to solution. Recent focus has been on examination of the feasibility of adapting the algorithm to enable triggering on events with highly-displaced vertices due to long-lived neutral particles that arise in new physics models such as Hidden Valley models and R-parity violating \SUSY. The trigger algorithm has been run on simulated Hidden Valley $H\rightarrow\pi_\mathrm{v}\pi_\mathrm{v}\rightarrow{b\bar{b}}{b\bar{b}}$ events to identify deficiencies and to explore distributions of relevant discriminant variables to optimise selection.\\
& \\
\firstcolumndata{Mar. 1998--} & {\bf {Assistant Research Scientist}}\\
\firstcolumndata{Sept. 2000} & {\it \htmladdnormallink{Centre for Time Metrology}{http://www.npl.co.uk/server.php?show=nav.348}, \htmladdnormallink{\NPL, UK}{http://www.npl.co.uk/}}\secondcolumndata{, 1998--2000}\\
%% & Assistant Research Scientist\\{\bf \htmladdnormallink{\NPL}{http://www.npl.co.uk/}}\\
& \\
& Provided technical and administrative support to a range of key activities relating to the maintenance and dissemination of the UK's national time scale.\\
%&\\
%& Developed software in Pascal and VBA to process data from \NPL's geodetic-quality GPS receivers. Collected and analysed GPS timing data. My work underpinned two projects, resulting in the publication of three time metrology research papers.\\%%\footnote{The results of this work are directly pertinent to long-baseline neutrino time-of-flight measurements.\vspace{-1.5\baselineskip}}.\\
% & Substantial contribution to three time metrology research papers.\\
%&\\
%& Maintained the caesium atomic clocks and time scale generation equipment that form the UK's national time scale. Compiled, processed and submitted data used to compute the international time scale (UTC). Operated a satellite earth station. Authorised changes (such as the start and end of daylight saving time) to the \htmladdnormallink{MSF}{http://www.npl.co.uk/server.php?show=ConWebDoc.998} 60\kHz national time radio signal, which synchronises radio-controlled clocks across the UK. Prepared technical manuals covering key procedures, in accordance with \NPL's ISO 9001 QA certification.\\
%&\\
%& Published monthly measurements bulletins and manned an enquiry service that received hundreds of telephone calls, letters and emails each year from the public.\\
%RF and Microwave Free Field Section (March 2000 - September 2000):
%I provided administrative support to NPL's antenna and field probe calibration service. This involved dealing with customers both face-to-face and over the telephone, preparing quotations, receiving and despatching customer's artefacts, scheduling calibration work, maintaining the contact database, and producing calibration certificates. I also assisted in the calibration of wire antennas used for Electromagnetic Compatibility (EMC) conformance testing.
%%&\\
%%& Substantial contribution to three time metrology research papers.\\
%research papers.\\
%% My work underpinned two projects, resulting in the publication of three papers~\cite{lowe:1998,lowe:1999,lowe:2000}.\\
%& \\
%\firstcolumndata{1996--1998} & \secondcolumndata{{\bf {Various temporary positions}}}\firstcolumndata{Various temporary positions}, including VCR retuning engineer for \htmladdnormallink{Channel 5 TV (UK)}{http://en.wikipedia.org/wiki/Five_tv}\secondcolumndata{, 1996--1998}\firstcolumndata{.}
\end{longtable}
